\documentclass{article}
\usepackage{geometry}
\usepackage{amsfonts}
\usepackage{amsmath}
\usepackage{enumitem}
\usepackage{titling}
\title{The Z Transform \\
\large A Crash Course}
\author{Brian Frost-LaPlante}
\date{Spring 2018}

\pretitle{\begin{center}\huge}
\posttitle{\end{center}}
\preauthor{\begin{center}\small}
\postauthor{\end{center}}
\predate{\begin{center}\footnotesize}
\postdate{\end{center}}
\setlength{\droptitle}{-40pt}

\begin{document}
\maketitle
\noindent This document is to serve as a supplementary document for students in ECE-210 who are not taking/have not taken ECE-211. It walks 
through only the concepts of discrete-time signal processing necessary for the scope of this course, and furthermore, understanding 
the applications of MATLAB in engineering industry.
\section*{Discrete-Time Signals}
\noindent Consider the space of real-valued functions with domain $\mathbb{Z}$. We call such functions \textit{discrete-time signals}, 
but don't get hung up on this name; the input is not necessarily a ``time'' quantity. To distinguish discrete-time signals from real-
valued functions in the usual sense, we denoted signal $x$ evaluated at integer $n$ by $x[n]$. Functions of this type are incredibly 
useful as, with some basic assumptions made, we can justly represent them with vectors in a computer programming language. \\ \\
\noindent This is a fairly important concept in computer simulation and data processing: computations made with computers cannot, 
realistically, involve irrational numbers, and as such, the representation of a real-valued function on a computer is quantized in two 
senses: the domain is discretized and the function value is quantized. The consideration of discrete-time signals is, in effect, the 
consideration of the first of these quantizations. We see this constantly in MATLAB; the use of the \textit{linspace} function to 
create an ``interval of the real line'' as a finite-length vector is exactly this kind of discretization. \\ \\
\noindent The processing of such signals by linear time-invariant (LTI) systems is of great interest, as these days, most designed 
systems will be running on FPGAs or some other quantized computing device. Put simply, a discrete-time LTI system takes a discrete-
time signal as an input and returns a discrete-time signal as an output, and is described by \textit{discrete-time convolution} with 
a discrete-time signal $h$. That is to say that a discrete-time LTI system is a mapping from discrete-time signals to discrete-time 
signals, and for input $x[n]$, the output $y[n]$ is given by
$$y[n] = \sum_{k=-\infty}^{\infty}x[k]h[n-k]$$
The signal $h$ is called the \textit{impulse response} of the system. Note that, if $x$ and $h$ are being represented as, say, 
vectors in MATLAB, this infinite sum is really a finite sum (as $x$ and $h$ are finite-length) and we make the assumption that they 
take value $0$ outside of the domain on which we have defined them. Convolution can be computed in MATLAB using the function 
\textit{conv}. Convolution is a binary operation, which maps two discrete-time signals to a discrete-time signal, and is commutative.
\section*{The Z Transform}
Those of you who have taken Differential Equations should recall a similar operation also called convolution for functions with domain 
$\mathbb{R}$. You will recall that this operation had a very special property in that its Laplace transform was simply the product of 
the Laplace transforms of the two argument functions. Presented here is the Z transform, which is analogous to the Laplace transform 
for signals with domain $\mathbb{Z}$. Given $x[n]$, a discrete-time signal, the Z transform of $x$ is a complex-valued function of 
a complex variable, $X$, given by 
$$X(z) = \sum_{n=-\infty}^{\infty}x[n]z^{-n}$$
should this series converge in some region of $\mathbb{C}$. Generally, this series may converge to different functions in different 
annular regions of convergence, and as such, when referring to \textit{the Z transform} of a signal, the region of convergence 
$R_1 < |z| < R_2$ should be specified. The Z transform is not usually computed by hand from this formula, but rather from some 
combination of known Z transforms and the properties it has. \\ \\
\noindent Why do we care about the Z transform though? One important property is that it maps convolutions to products. That is to say 
that for an LTI system with impulse response $h$, for each input $x$, the output $y$ has Z transform
$$Y(z) = X(z)H(z)$$
In many cases, this is much simpler to compute than a convolution. We also can gain a lot of insight about a system by looking at 
the Z transform of its impulse response, $H(z)$. Namely, we take interest in the locations of its poles: points at which it is not 
defined. The regions of convergence of $H$ are generally annular rings whose inner and outer radii are determined by the positions 
of poles in the complex plane. A region of convergence contains no poles. The MATLAB function \textit{zplane} makes it easy to see 
where these poles lie, and thereby where the regions of convergence lie. We say a system is stable (a bounded input yields a bounded 
output) if the region of convergence contains the unit circle, $|z| = 1$. We say a system is causal if the region of convergence 
contains the point at $\infty$, i.e. there is no pole at $\infty$. Causality simply means that, in a physical sense, the output of 
the system will not depend on future outputs of the same system.
\end{document}
