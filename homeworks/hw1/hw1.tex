\documentclass[11pt]{article}

\usepackage{amsmath,amssymb,graphicx,enumerate}
\usepackage{hyperref}
\usepackage[parfill]{parskip}
\hypersetup{
    colorlinks=true,
    linkcolor=blue,
    filecolor=magenta,      
    urlcolor=blue,
}

\def\Homework{1} % Number of Homework
\def\Session{Spring 2020}
\def\Section{B}
\def\MyEmail{guybymatlab@gmail.com}
\def\DateOfSubmission{January 28rd }

\title{MATLAB Assignment \Homework}
\author{\Session, Section \Section}
\date{}

\newenvironment{qparts}{\begin{enumerate}[{(}a{)}]}{\end{enumerate}}

\textheight=9in
\textwidth=6.5in
\topmargin=-.75in
\oddsidemargin=0.25in
\evensidemargin=0.25in


\begin{document}
\maketitle
This homework is designed to get you used to thinking in terms of
matrices and vectors, as this is how MATLAB stores its data. 
You will find that complicated operations can be resolved by one 
to two lines of code if you use the correct function and store data 
in the appropriate form.
The other purpose of this homework is to get you comfortable
with using the \textbf{\textit{help}} and \textbf{\textit{doc}} functions.

As with all the homeworks, please submit it as a \textit{.m} file, 
with suppressed output.
Remember that all lectures and homeworks may be found at 
\textit{github.com/guybaryosef/ECE210-materials}.
This homewis due by 11:59 PM on \DateOfSubmission to \MyEmail.
Remember to bring a hardcopy in to next class! 
\newline

\noindent \textbf{1. Creating Scalar Variables:} 
Create the following variables. 
Each construction should be done in \textbf{one} line.
Please use the assigned variable names. 
\begin{qparts}
\item
$ a = \frac{5.7 \pi}{6.9} $
\item 
$ b = 239+e^5 - 2.5 \times 10^{23}$
\item
$ c = ln(4.23) \times sin^{-1}(0.7)$
\item 
$ z = (3+2j) \times (4+5j) $

\end{qparts}

\noindent 
\newline
\textbf{2. Complex Operations} 
Find the real part, imaginary part, magnitude, 
phase and complex conjugate of $z$ calculated in question 1(d),
choosing appropriate variable names for each.

\noindent 
\newline
\textbf{3. Vector and Matrix Variables} 
Create the following variables. 
Make sure to use the assigned variable names. 
When doing parts c and d, 
make sure you know when to use the colon operator \textbf{\textit{:}},
and when to use \textbf{\textit{linspace}}. 

\begin{qparts}
    \item Create a row vector where 
    $ aVec = \begin{bmatrix}3.14&15&9&26+0.1j\end{bmatrix}$,
    and with it generate two matrices, $A1$ using \textbf{\textit{repmat}}
    and $A2$ using concatenation, such that:
    \begin{align*}
    A1 = A2 = \begin{bmatrix}3.14&15&9& 26+0.1j \\ 3.14&15&9&26+0.1j\\3.14&15&9&26+0.1j\end{bmatrix}
    \end{align*}

    \item Create two versions of the column version of $aVec$,
    using the matrix operation $.'$ and
    \textbf{\textit{transpose}} function in MATLAB.
    Name the variables $bVec1$ and $bVec2$ respectively.

    \item Create a row Vector $cVec$ where the numbers range from -5 to 5
    in increasing order and with an interval of 0.1 between consecutive numbers.

    \item Create a column vector $dVec$ with 100 evenly spaced points between -5 and 5.
    Do not use the same operator as in part c.

    \item Create a matrix $A$ where 
    $A = \begin{bmatrix} 1+2j&10^{-5}\\ e^{j2\pi}&3+4j \end{bmatrix}$.

    \item Use \textbf{\textit{magic}} and divide by 65 to create a $5 \times 5$
    doubly stochastic matrix $B$.

    \item Create a $5 \times 5$ matrix, $C$,
    such that each element is drawn from the standard normal distribution.
    (Note: You'll need to look up how to make it).

\end{qparts}

\noindent 
\newline
\textbf{4. Vector and Matrix Operations}
Using the variables made in question 3,
perform the following operations:

\begin{qparts}

    \item  Compute the dot product between $aVec$ and $bVec$, using both the
    \textbf{\textit{dot}} function as well as element-wise multiplication
    followed by the \textbf{\textit{sum}} function.
    Label the results $d1$ and $d2$ respectively.

    \item Compute $E = BC$.

    \item Compute 
    $G = \frac{1}{4}A^3 + \frac{1}{4}A^2 + \frac{1}{3}A + \frac{1}{6}I$. 

    \item Compute $H = A^{-1}$.

    \item Save the dimensions of $cVec$ and $dVec$ as the variables $cVec\_dim$ and
    $dVec\_dim$, respectively.

\end{qparts}

\end{document}

