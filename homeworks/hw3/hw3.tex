\documentclass[11pt]{article}
\usepackage{amsmath,amssymb,graphicx,enumerate}
\usepackage{hyperref}
\usepackage[parfill]{parskip}
\hypersetup{
    colorlinks=true,
    linkcolor=blue,
    filecolor=magenta,      
    urlcolor=blue,
}

\def\Homework{3} % Number of Homework
\def\Session{Spring 2019}
\def\Section{B}
\def\MyEmail{guybymatlab@gmail.com}
\def\DateOfSubmission{February 6th }

\title{MATLAB Assignment \Homework}
\author{\Session, Section \Section}
\date{}

\newenvironment{qparts}{\begin{enumerate}[{(}a{)}]}{\end{enumerate}}

\textheight=9in
\textwidth=6.5in
\topmargin=-.75in
\oddsidemargin=0.25in
\evensidemargin=0.25in


\begin{document}
\maketitle
This problem set's goal is to familiarize you with two useful concepts in MATLAB,
functions and logical indexing, as well as begin getting you used to plotting.

Please submit this homework as \textit{.m} files (note the plural), 
with suppressed output (obviously, the plots will still be displayed).
Remember that all lectures and homeworks may be found at 
\textit{github.com/guybaryosef/ECE210-materials}.
This homework is due at 11:59PM on \DateOfSubmission to \MyEmail. 


\noindent
\newline
\textbf{1. Lunar Eclipse} 
This question guides you through some basic image processing techniques in MATLAB. 
You will create interesting images with relational and logical indexing, 
as well as use the \textbf{\textit{imshow}} function to visualize what you have created.  
\begin{enumerate}[a.]

    \item Create a $100 \times 100$ matrix $A$ whose contents are all ones.
    
    \item Create a $100 \times 100$ matrix $B$ whose contents are all zeros.
    
    \item In matrix $A$, set the values of entry $a_{i,j}$ equal to 0 
    if $\sqrt{(i-50)^2 + (j-50)^2} < 20$. 
    \textbf{Hint} : \textbf{\textit{meshgrid}} would be useful in creating the indices.
    
    \item In matrix $B$, 
    set the values of entry $a_{i,j}$ equal to 1 if $\sqrt{(i-40)^2 + (j-40)^2} < 20$.
    
    \item Visualize the following results with \textbf{\textit{figure}} and \textbf{\textit{imshow}}. 
    Describe each of the results with one sentence each,
    either as a comment or by printing to stdout. 
    \begin{itemize}
        \item $A$
        \item $B$
        \item The intersection between $A$ and $B$
        \item The union between $A$ and $B$
        \item Complement of the intersection between $A$ and $B$
        \item Complement of the union between $A$ and $B$
    \end{itemize}

\end{enumerate}


\noindent
\newline
\textbf{2. Sincing Ship} 
Here we will look at a function near and dear to the signal processing community's heart -
the sinc function.
You are going to implement your own functions to find the local extrema and roots.
As we haven't yet gone over plotting,
don't stress over stylizing your plots beyond what is asked.

\begin{enumerate}[a.]
    \item Sample a sinc with 10001 linearly spaced points on $[-2\pi, 2\pi]$ using 
    the \textbf{\textit{sinc}} function. Plot your results, and give the plot a title.

    \item Create a function (either anonymously or in another file) which locates
    the indices at which the input vector transitions from one sign to another.
    \textbf{Note:} This can be done in one line of code but it is $trecherous^{FF}$.
    For one scenario the vector has a positive value and then a negative value,
    i.e. $v(n) > 0$ and $v(n+1) < 0$. 
    The root occurs somewhere in between, you can pick either $n$ or $n+1$.
    We could loop through and check this condition at every point - don't do that.
    Instead think of a way to use logical indexing:
    You will want to write conditions on the vector and some kind of shifted version of itself.
    Beware however, when you do this you will have non-overlapping points.
    It is up to you to figure out what to with them.

    \item Apply your function to the sinc you created.
    Find the roots (x and y coordinates) and plot them as black circles
    on top of the sinc using \textbf{\textit{plot(xRoots,yRoots,'ko')}}. 

    \item Now we are interested in finding the extrema
    (local minimums and maximums).
    Firstly, approximate the derivative by taking the difference between all adjacent elements
    and dividing by their time spacing.
    Then apply your function to the approximate derivative of your sinc to obtain the extrama.
    Finally, plot them as red stars on top of the sinc using \textbf{\textit{plot(xMinMax,yMinMax,'r*')}}.
\end{enumerate}


\noindent
\newline
\textbf{3. Fun with find}
Write a function to return the value and index of a number in a 
vector / matrix that is closest to a desired value.
The function should be called as $[val, ind] = findClosest(x, desiredValue)$.
This function can be accomplished in less than five lines.
Show that the function works by finding the value closest to $3/2$
(and index of said value) in \textbf{\textit{sin( linspace(0,5,100) ) + 1}}.
You may find the functions \textbf{\textit{abs}}, \textbf{\textit{min}} and/or \textbf{\textit{find}} useful.
\textbf{Hint}: You may have some trouble using \textbf{\textit{min}} when \textit{x} is a matrix-
remember that you could convert any matrix to a vector using $y = X(:)$.


\end{document}
