\documentclass[11pt]{article}
\usepackage{amsmath,amssymb,graphicx,enumerate}
\usepackage{hyperref}
\usepackage[parfill]{parskip}
\hypersetup{
    colorlinks=true,
    linkcolor=blue,
    filecolor=magenta,      
    urlcolor=blue,
}

\def\Homework{6} % Number of Homework
\def\Session{Spring 2019}
\def\Section{B}
\def\MyEmail{guybymatlab@gmail.com}
\def\DateOfSubmission{March 20th }

\title{MATLAB Assignment \Homework}
\author{\Session, Section \Section}
\date{}

\newenvironment{qparts}{\begin{enumerate}[{(}a{)}]}{\end{enumerate}}

\textheight=9in
\textwidth=6.5in
\topmargin=-.75in
\oddsidemargin=0.25in
\evensidemargin=0.25in


\begin{document}
\maketitle
This homework deals with digital filters in a low-level sense.
You are expected to know a bit about the z-transform,
but if you are not in Signals and Systems,
please contact me separately for some additional information on this homework.
All plots should have a title, x-axis and y-axis labels,
and if there is more than one function in the same figure, a legend as well.
Additionally, make sure your plot's axis bounds are adequate. 

Please submit this homework as a \textit{.m} file, 
with suppressed output.
Remember that all lectures and homeworks may be found at 
\textit{github.com/guybaryosef/ECE210-materials}.
Homework is due on \DateOfSubmission to \MyEmail.

\noindent
\newline
\textbf{1. Z it up!}
For this question,
you will be working with the discrete system described by the transfer function: 
$$H(z) = \frac{\frac{1}{2}+\frac{2}{3}z+\frac{3}{7}z^2}{2+\frac{1}{3}z+\frac{1}{2}z^3}$$
\begin{enumerate}[a.]
    \item Store this transfer function in MATLAB as numerator and denominator vectors and
    then find the poles and zeros.
    
    \item Plot the poles and zeros of $H$.

    \item Use \textit{\textbf{impz}} to obtain the first 100 points of the impulse response
    and plot them using the appropriate plotting function. 
    
    \item Let $x[n] = (-\frac{3}{4})^n$, and take n from 0 to 99.
    Apply the digital filter $H$ to $x$ using \textit{\textbf{filter}}. 
    
    \item Now let us apply the filter analytically using convolution.
    Apply the digital filter to \textit{x} using \textit{\textbf{conv}}.
    
    \item Plot your results for part (d) and (e) in 2 side-by-side subplots in order to
    show that they are equivalent. Note that you will have to throw out some values from
    part (e) to get the same result.


\end{enumerate}

\noindent
\newline
\textbf{2. You Gotta be Fibbin' Me!}
The Fibonacci sequence is a sequence of numbers such that every number after the first two
is the sum of the two preceding numbers. The first two numbers of the sequence are 0 and 1.
It is cute to imagine a discrete-tyme system whose impulse response is the Fibonacci sequence!

\begin{enumerate}[a.]
    \item Use a for loop to generate the first 100 values of the Fibonacci sequence and plot 
    these values using MATLAB and plot them using \textit{\textbf{semilogy}}.

    \item Assuming this is the impulse response of a system,
    find the output of the system with input $x[n]$ from the previous problem and plot it 
    using an appropriate function.
\end{enumerate}

\end{document}