\documentclass[11pt]{article}
\usepackage{amsmath,amssymb,graphicx,enumerate}
\usepackage{hyperref}
\usepackage[parfill]{parskip}
\hypersetup{
    colorlinks=true,
    linkcolor=blue,
    filecolor=magenta,      
    urlcolor=blue,
}

\def\Homework{9} % Number of Homework
\def\Session{Spring 2019}
\def\Section{B}
\def\MyEmail{guybymatlab@gmail.com}

\title{MATLAB Assignment \Homework}
\author{\Session, Section \Section}
\date{}

\newenvironment{qparts}{\begin{enumerate}[{(}a{)}]}{\end{enumerate}}

\textheight=9in
\textwidth=6.5in
\topmargin=-.75in
\oddsidemargin=0.25in
\evensidemargin=0.25in


\begin{document}
\maketitle
Please submit this homework as a \textit{.m} file, 
with suppressed output.
Remember that all lectures and homeworks may be found at 
\textit{github.com/guybaryosef/ECE210-materials}.
Homework is due by the last day of the semester to \MyEmail.

\noindent
\newline
\textbf{1. Dipping your toes:}
For this problem you will need to download a \textit{csv} file that
contains a time series, for which you will build an AR model and then
use it for forecasting. Remember that an AR model has the equation:
\begin{eqnarray}\label{eq:AR-Model}
    X(t) = \delta + \sum_{n = 1}^{p}\alpha_nr(t-n) + v_t
\end{eqnarray}

\begin{enumerate}[a.]
    \item A \textit{csv} file, which stands for 'comma separated variable',
    is a file format used to store data into a spreadsheet, 
    with the columns separated by commas and the rows separated by 
    newline characters.
    To begin this problem, download and import the file called \textit{GOOG.csv}
    from:
    \textit{github.com/guybaryosef/ECE210-materials/tree/master/homeworks/hw9}.
    In this file you will find the adjusted closing value of Alphabet's stock
    (GOOG) over the past year.
    For simplicity, you can ignore the date associated with closing value
    (note that weekends and certain holidays will result in skipped days),
    and instead we can treat each day as an equal-distanced time step.

    \item Run a least squares fit on the time series to find the AR
    coefficients of an AR model of order 10 (\textbf{Hint}: You can either check
    out MATLAB's built-in functions or look back at lesson 10 for a way to code
    this).
    Notice that to be able to do the linear regression, you will need to
    start 10 steps into the time series,
    thereby shrinking it by 10. 

    \item Using the AR model found in the previous section, use equation
    \eqref{eq:AR-Model} to forecast the next 10 days of activity.
    Unfortunately, if you look at the actual stock over those future 10 days,
    you will see that your forecasting was... slightly less than accurate. 
    Not to fret!
    This is just an example; a chance to dip your toes into one part of the
    of quantitative finance.
\end{enumerate}

\noindent
\newline
\textbf{2. Now walk it out...}
You can think of a random walk as a running (cumulative) sum of
i.i.d. random variables. In our 1-dimensional case,
the next value in the series is the last value plus or minus 1, each with 
a 50\% chance.
    
\begin{enumerate}[a.]
    \item  Generate 50 random walks with the properties described above,
    each with a total of 300 steps.
    
    \item  Plot the random walks you generated all on top of one another.
    There is no need for a plot legend.
    
    \item  Now repeat parts (a) and (b) for a random walk where the
    probability of $+1$ is $75\%$ and the probability of $-1$ is $25\%$.
    Is there a difference between these results and your previous plot?
\end{enumerate}

\end{document}