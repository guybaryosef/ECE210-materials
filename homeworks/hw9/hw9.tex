\documentclass[11pt]{article}
\usepackage{amsmath,amssymb,graphicx,enumerate}
\usepackage{hyperref}
\usepackage[parfill]{parskip}
\hypersetup{
    colorlinks=true,
    linkcolor=blue,
    filecolor=magenta,      
    urlcolor=blue,
}

\def\Homework{9} % Number of Homework
\def\Session{Spring 2019}
\def\Section{B}
\def\MyEmail{guybymatlab@gmail.com}
\def\DateOfSubmission{ --------- }

\title{MATLAB Assignment \Homework}
\author{\Session, Section \Section}
\date{}

\newenvironment{qparts}{\begin{enumerate}[{(}a{)}]}{\end{enumerate}}

\textheight=9in
\textwidth=6.5in
\topmargin=-.75in
\oddsidemargin=0.25in
\evensidemargin=0.25in


\begin{document}
\maketitle
Please submit this homework as a \textit{.m} file, 
with suppressed output.
Remember that all lectures and homeworks may be found at 
\textit{github.com/guybaryosef/ECE210-materials}.
Homework is due on \DateOfSubmission to \MyEmail.

\noindent
\newline
\textbf{1. Math or just dumb?}
For this problem you will need to download a \textit{csv} file that
contains a time series, for which you will build an AR model and then
use it for forecasting. Remember that an AR model has the equation:
\begin{eqnarray}\label{eq:AR-Model}
    X(t) = \delta + \sum_{n = 1}^{p}\alpha_nr(t-n) + v_t
\end{eqnarray}

\begin{enumerate}[a.]
    \item Download the \textit{csv} file located at 
    \textit{github.com/guybaryosef/ECE210-materials/hw/hw9/SnP500-2010.csv}.
    A \textit{csv} file, or 'comma separated variable' file,
    is a file format used to store data into a spreadsheet, 
    with columns separated by commas and rows separated by 
    newline characters.
    In this file you will find the closing value of the S\&P 500 over every
    'work' day for the year 2010.
    For simplicity, you can ignore the date associated with closing value
    (as weekends and certain holidays will result in skipped days), and we 
    can simply look at each closing value as the next 'step', all of even sizes.

    \item Run a least squares fit on the time series to find the AR
    coefficients of an AR model of order 5 (\textbf{Hint}: Look at lesson
    10 for how to approach this part).

    \item Using the AR model found in the previous section, use equation
    \eqref{eq:AR-Model} to forecast the next 10 days.
\end{enumerate}

\noindent
\newline
\textbf{2. Now walk it out...}
You can think of a random walk as a running (cumulative) sum of
i.i.d. random variables. In our case 1-dimensional case,
the next value in the series is the last value plus or minus one with a 50\%
chance. This can be written mathematically as:
\begin{eqnarray*}
    X[0] = 0 \\
    X[t+1] = X[t] + \begin{cases} 
            \delta_1, & p \\
            \delta_2, & 100\leq x 
        \end{cases}
\end{eqnarray*}
    
\begin{enumerate}[a.]
    \item  Generate 50 random walks with the properties described above, with
    a total of 300 steps.
    
    \item  Plot the random walks you generated all on top of one another.
    
    \item  Now repeat parts (a) and (b) for a random walk where the
    probability of +1 is 75\% and the probability of-1 is 25\%.
    What is the difference between these results and your previous plot?
\end{enumerate}

\end{document}