\documentclass[11pt]{article}
\usepackage{amsmath,amssymb,graphicx,enumerate}
\usepackage{hyperref}
\usepackage[parfill]{parskip}
\hypersetup{
    colorlinks=true,
    linkcolor=blue,
    filecolor=magenta,      
    urlcolor=blue,
}

\def\Homework{8} % Number of Homework
\def\Session{Spring 2019}
\def\Section{B}
\def\MyEmail{guybymatlab@gmail.com}
\def\DateOfSubmission{ --------- }

\title{MATLAB Assignment \Homework}
\author{\Session, Section \Section}
\date{}

\newenvironment{qparts}{\begin{enumerate}[{(}a{)}]}{\end{enumerate}}

\textheight=9in
\textwidth=6.5in
\topmargin=-.75in
\oddsidemargin=0.25in
\evensidemargin=0.25in


\begin{document}
\maketitle
Please submit this homework as a \textit{.m} file, 
with suppressed output.
Remember that all lectures and homeworks may be found at 
\textit{github.com/guybaryosef/ECE210-materials}.
Homework is due on \DateOfSubmission to \MyEmail.


\noindent
\newline
\textbf{1. State of the Space}


\noindent
\newline
\textbf{2. Girl look at that Bode!}
A continuous time system is characterized by the following equation:
\begin{align*}
    f=t^5e^{−2t}sin(5t) +t^3e^{−3}tcos(4t) +tcos(10t)
\end{align*}

\begin{enumerate}[a.]
    \item Take the Laplace Transform of the above equation. 
    (Hint:  Use the symbolic toolbox!)
    
    \item Use \textbf{\textit{numden}} and \textbf{\textit{coeffs}}
    to extract the coefficients from the Laplace transform. 
    Use \textbf{\textit{double}} to convert them to numeric form.
    
    \item Make a bode plot for the equation you obtained above. 
    Turn the grid on. Why does the grid look the way it does?
\end{enumerate}

\end{document}