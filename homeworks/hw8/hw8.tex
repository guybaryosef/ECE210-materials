\documentclass[11pt]{article}
\usepackage{amsmath,amssymb,graphicx,enumerate}
\usepackage{hyperref}
\usepackage[parfill]{parskip}
\hypersetup{
    colorlinks=true,
    linkcolor=blue,
    filecolor=magenta,      
    urlcolor=blue,
}

\def\Homework{8} % Number of Homework
\def\Session{Spring 2019}
\def\Section{B}
\def\MyEmail{guybymatlab@gmail.com}

\title{MATLAB Assignment \Homework}
\author{\Session, Section \Section}
\date{}

\newenvironment{qparts}{\begin{enumerate}[{(}a{)}]}{\end{enumerate}}

\textheight=9in
\textwidth=6.5in
\topmargin=-.75in
\oddsidemargin=0.25in
\evensidemargin=0.25in


\begin{document}
\maketitle
Please submit this homework as a \textit{.m} file, 
with suppressed output.
Remember that all lectures and homeworks may be found at 
\textit{github.com/guybaryosef/ECE210-materials}.
Homework is due by the last day of the semester to \MyEmail.


\noindent
\newline
\textbf{1. One plus one makes...}
The fourier transform allows us to transform signals back and 
forth between the time and frequency domains.
While this can be relatively understood for time series signals
(such as sound, for example), what this means for other forms
of data might be more unclear.
In this problem we will create a cool hybrid picture phenomena that
is possible through the fourier transform and hopefully as we go 
through the different parts, you will get a sense of what the frequency
domain means for images.
If you are in need of a reference,
much of this question is similar to the image part shown in lesson 7.

\begin{enumerate}[a.]
    \item To begin, please download the two \textit{.png} images located at
    \textit{github.com/guybaryosef/ECE210-materials/tree/master/homeworks/hw8}
    and import them into your MATLAB script file.

    \item Notice that these images are not the same size!
    As a preprocessing step, shrink the images so that they have the same
    pixel dimensions.
    Note that to do this, you do not want to just delete the first X pixels 
    from the larger picture, because this will un-center the image.
    Instead if one picture is, for example, 50 pixels larger than the other
    along the x axis, remove its first and last 25 pixels along the x axis,
    that way equalizing its size while still keeping it centered.
    At this point, also convert your pixel values into a double type and 
    normalize the values.
    Plot the two images beside one another using \textit{\textbf{subplot}}
    and \textit{\textbf{imshow}}.
    You should currently be seeing the fab-tastic faces of Marilyn Monroe and
    Albert Einstein looking out at you.

    \item Next we will like to create a 2-d lowpass filter.
    One popular option is called a gaussian filter function,
    which unsurprisingly looks like a 3-d guassian distribution.
    The equation for this filter is:
    \begin{align*}
        g(x,y) = \exp{-\frac{(x-a)^2 +(y-b)^2}{2\sigma^2}}
    \end{align*}
    where $\sigma$ is the standard deviation and $(a,b)$ is the center of the function.
    Using a standard deviation of $14$,
    create this filter so that it has the same dimensions as the two images
    (\textbf{Hint:} Use \textbf{\textit{meshgrid}}).
    Note that you want the center of the filter, $(a,b)$, to be the center of the image.
    Quickly plot your results using \textbf{\textit{surf}} (titles and the like unnecessary),
    so that you can see how this resembles a lowpass filter in the frequency domain.
    If your graph is black, you should make your graph a variable and add
    the following line after it:
    \textbf{\textit{set(your-graph's-variable-name, 'LineStyle', 'none')}}.

    \item Just as the above can be a lowpass filter, we can quickly make
    a highpass filter by calculating $g2(x,h) = 1 - g(x,y)$.
    Do so here.

    \item Take the fast fourier transform of each image using
    \textit{\textbf{fftshift(fft2(img))}}.
    Lets now apply the filters to the images.
    Because we want to combine these two images together,
    let's emphasize for each image its characterstics.
    What I mean here is that Marilyn has smooth, silky hair,
    and so a lowpass filter, which has the effect of \textit{smoothing}
    an image, is appropriate.
    In the same vain, Albert's image has a lot more edges, creases, and lines to it 
    (I will avoid saying anything meaner) and so you should apply the highpass filter
    to it.
    Remember that convolution in the time domain is multiplication in the frequency domain
    and so all that is necessary here is element-wise multiplication.
    Create another subplot, this one 2x2, with the original images in the frequency domain
    as well as the filtered images in the frequency domain.

    \item Using the inverse fft,
    return each image back to the time domain, and subplot them side by side.
    As a bonus question, where is Albert?

    \item Average the two images together, and you now have a hybrid photo!
    Gah! A little scary at first sight!
    However look at it up close. Now back away from the screen and look at it again.
    Do you see different faces peering up at you?
    If not, feel free to tweak around with the standard deviation of the filters as 
    well as the other parameters.
    This question was taken directly from 
    \textit{https://jeremykun.com/2014/09/29/hybrid-images/},
    which is an awesome blog full of cool and informative ideas and concepts. 
\end{enumerate}


\noindent
\newline
\textbf{2. Girl look at that Bode!}
A continuous time system is characterized by the following equation:
\begin{align*}
    f=t^5e^{−2t}sin(5t) +t^3e^{−3}tcos(4t) +tcos(10t)
\end{align*}

\begin{enumerate}[a.]
    \item Take the Laplace Transform of the above equation. 
    (Hint: Use the symbolic toolbox!)
    
    \item Use \textbf{\textit{numden}} and \textbf{\textit{coeffs}}
    to extract the coefficients from the Laplace transform. 
    Use \textbf{\textit{double}} to convert them to numeric form.
    
    \item Create a bode plot for the equation you obtained above. 
    As is customary with log-scaled axis (to differentiate from the
    more standard linear spacing), turn the grid on.
\end{enumerate}

\end{document}