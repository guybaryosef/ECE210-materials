\documentclass[11pt]{article}
\usepackage{amsmath,amssymb,graphicx,enumerate}
\usepackage{hyperref}
\usepackage[parfill]{parskip}
\hypersetup{
    colorlinks=true,
    linkcolor=blue,
    filecolor=magenta,      
    urlcolor=blue,
}

\def\Homework{7} % Number of Homework
\def\Session{Spring 2020}
\def\Section{A}
\def\MyEmail{guybymatlab@gmail.com}

\title{MATLAB Assignment \Homework}
\author{\Session, Section \Section}
\date{}

\newenvironment{qparts}{\begin{enumerate}[{(}a{)}]}{\end{enumerate}}

\textheight=9in
\textwidth=6.5in
\topmargin=-.75in
\oddsidemargin=0.25in
\evensidemargin=0.25in

\begin{document}
\maketitle
In this assignment, you will reinforce what we did lecture 7
regarding MATLAB's filter toolbox.
Please submit this homework as a \textit{.m} file, 
with suppressed output.
Remember that all lectures and homeworks may be found at 
\textit{github.com/guybaryosef/ECE210-materials}.
Homework is due by the end of the semester to \MyEmail.

For each of the following questions,
generate filters using either \textbf{\textit{filterDesigner}}
 or the filter design toolbox in the DSP System toolbox.
Apply the filter to the signal using \textit{\textbf{step}} or \textit{\textbf{filter}},
depending on how your filter is represented.
Lastly, plot the Fourier Transform of the final result using \textit{\textbf{fft}}, \textit{\textbf{fftshift}},  
and \textit{\textbf{plot}}.
Refer to the notes for the proper way to use \textit{\textbf{fft}} and obtain the proper scaling.

1. Generate a signal that consists of a sum of sine waves of frequencies ranging from 1 to 50 kHz.
Set $t$ to be from 0 to 2 seconds, using an interval of $0.001s$:

$$ signal = \sum_{f=1}^{50000} sin(2\pi ft)$$

2. Create a Butterworth lowpass filter with a sampling frequency of $F_s = 100 kHz$,
a passband frequency of $F_{pass} = 10 kHz$, a stopband frequency of $F_{stop} = 20 kHz$,
a passband attenuation of $A_{pass} = 5dB$, and a stopband attenuation of $A_{stop} = 50dB$.

3. Create a Chebychev I highpass filter with a sampling frequency of $F_s = 100 kHz$,
a passband frequency of $F_{pass} = 35 kHz$, a stopband frequency of $F_{stop} = 15 kHz$,
a passband attenuation of $A_{pass} = 2dB$, and a stopband attenuation of $A_{stop} = 40dB$.

4. Create a Chebychev II bandstop filter with a sampling frequency of $F_s = 100 kHz$,
a passband frequency of below the frequency $F_{pass1} = 5 kHz$ and above $F_{pass2} = 45 kHz$,
a stopband frequency of between $F_{stop1} = 15 kHz$ and $F_{stop2} = 35kHz$,
a passband attenuation of Apass = 5dB, and a stopband attenuation of Astop = 50dB.

4. Create a Elliptic bandpass filter with a sampling frequency of $F_s = 100 kHz$,
a stopband frequency of below the frequency $F_{stop1} = 15 kHz$ and above $F_{stop2} = 35 kHz$,
a passband frequency of between $F_{pass1} = 20 kHz$ and $F_{pass2} = 30 kHz$,
a passband attenuation of $A_{pass} = 5dB$, and a stopband attenuation of $A_{stop} = 50dB$.

\end{document}