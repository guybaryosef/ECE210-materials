\documentclass[11pt]{article}
\usepackage{amsmath,amssymb,graphicx,enumerate}
\usepackage{hyperref}
\usepackage[parfill]{parskip}
\hypersetup{
    colorlinks=true,
    linkcolor=blue,
    filecolor=magenta,      
    urlcolor=blue,
}

\def\Homework{2} % Number of Homework
\def\Session{Spring 2020}
\def\Section{A}
\def\MyEmail{guybymatlab@gmail.com}
\def\DateOfSubmission{February 5th }

\title{MATLAB Assignment \Homework}
\author{\Session, Section \Section}
\date{}

\newenvironment{qparts}{\begin{enumerate}[{(}a{)}]}{\end{enumerate}}

\textheight=9in
\textwidth=6.5in
\topmargin=-.75in
\oddsidemargin=0.25in
\evensidemargin=0.25in


\begin{document}
\maketitle
This problem set will cement your understanding of vector operations and
go over several important built-in functions.
It will also provide an example of the efficiency to be gained by pre-allocating
your data structures.

As with all the homeworks, please submit it as a \textit{.m} file, 
with suppressed output.
Remember that all lectures and homeworks may be found at 
\textit{github.com/guybaryosef/ECE210-materials}.
This homework is due by 4:00 PM on \DateOfSubmission to \MyEmail.
Remember to also bring a hardcopy in to class! 

\noindent
\newline
\textbf{1. Vector? I hardly know her!}
Here we will look at some applications of built-in vectorized functions.
Label your variables appropriately.
\begin{qparts}
    \item Create a vector of 100 evenly spaced samples of
    the exponential function over the interval [0,1].

    \item Approximate the integral using the trapezoidal method
    (use \textbf{\textit{trapz}} and multiply by the interval)
    as well as the rectangular method (sum over all points and multiply by the interval).

    \item Approximate the cumulative integral using the trapezoidal method
    (use \textbf{\textit{cumtrapz}}) and the rectangular method
    (use \textbf{\textit{cumsum}}).
    Looking at the pair of cumulative values,
    did you get the same answers as in part (b)?

    \item Approximate the derivative by taking the difference between
    all adjacent elements and dividing by the time spacing.
    Similarly, approximate the second derivative.
    What are the lengths of each derivative vector?

\end{qparts}

\noindent
\newline
\textbf{2. Array Foray} Perform the following matrix operations. 
\begin{qparts}
    \item Use \textbf{\textit{reshape}} to create a $10 \times 10$ matrix $A$ where 
    $A = \begin{bmatrix}1 &11 & ...& 91\\ 2&12&...&92\\ \vdots&\vdots&\ddots&\vdots\\ 10&20&...&100\end{bmatrix}$.

    \item Use \textbf{\textit{magic}} to create a $10 \times 10$ magic matrix $B$. 
    Use $B$ to create a matrix $C$ which has the same diagonal values of $B$ and is zero elsewhere. 
    \textbf{Note}: You might want to look up \textbf{\textit{diag}} to see how to do this elegantly. 

    \item Flip the second column of $B$ such that it is inverted upside down.

    \item Flip the matrix $A$ from left to right.

    \item Find the column-wise sum of every column of $AB$ (normal matrix multiplication).
    The result should be a row vector.

    \item Find the row-wise mean of every row of $AB$ (element-wise matrix multiplication).
    The result should be a column vector.

    \item Delete the last column of $A$. 
\end{qparts}

\noindent
\newline
\textbf{3. Gotta Go Fast}
Generate a $300 \times 500$ matrix with entries $a_{i,j} = \frac{i^2+j^2}{i+j+3} $
using the following methods and use \textbf{\textit{tic toc}} to time the speed of each.
Report the times in a table (using the \textbf{\textit{table}} function).

\begin{qparts}
    \item Using \textit{for} loops and no pre-allocation.

    \item Using \textit{for} loops and pre-allocating memory with \textbf{\textit{zeros}}.

    \item Using only element-wise matrix operations.
    \textbf{Note}: \textbf{\textit{repmat}} and/or \textbf{\textit{meshgrid}} will be useful here. 
\end{qparts}

\end{document}
