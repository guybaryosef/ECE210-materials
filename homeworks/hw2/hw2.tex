\documentclass[11pt]{article}
\usepackage{amsmath,amssymb,graphicx,enumerate}
\usepackage{hyperref}
\usepackage[parfill]{parskip}
\hypersetup{
    colorlinks=true,
    linkcolor=blue,
    filecolor=magenta,      
    urlcolor=blue,
}

\def\Homework{2} % Number of Homework
\def\Session{Spring 2019}
\def\Section{B}
\def\MyEmail{guybymatlab@gmail.com}
\def\DateOfSubmission{ --------- }

\title{MATLAB Assignment \Homework}
\author{\Session, Section \Section}
\date{}

\newenvironment{qparts}{\begin{enumerate}[{(}a{)}]}{\end{enumerate}}

\textheight=9in
\textwidth=6.5in
\topmargin=-.75in
\oddsidemargin=0.25in
\evensidemargin=0.25in


\begin{document}
\maketitle
This problem set will cement your understanding of array operations and go over 
several important built in functions.
It also explores relational and logical indexing of matrices as well as basic plotting in MATLAB.

As with all the homeworks, please submit it as a \textit{.m} file, 
with suppressed output.
Remember that all lectures and homeworks may be found at 
\textit{github.com/guybaryosef/ECE210-materials}.
Homework is due on \DateOfSubmission to \MyEmail. 

\noindent
\newline
\textbf{1. Lunar Eclipse} 
This question guides you through some basic image processing techniques in MATLAB.
You will create interesting images with relational and logical indexing,
as well as the \emph{imshow} function to visualize what you have created. 

\begin{itemize}
    \item Create a $100 \times 100$ $A$ where its contents are all ones.

    \item Create a $100 \times 100$ $B$ where its contents are all zeros.

    \item In matrix $A$, set the values of entry $a_{i,j}$ equal to 0 if $\sqrt{(i-50)^2 + (j-50)^2} < 20$.
    \textbf{Hint} : \textbf{\textit{meshgrid}} would be useful in creating the indices.

    \item In matrix $B$, set the values of entry $a_{i,j}$ equal to 1 if $\sqrt{(i-40)^2 + (j-40)^2} < 20$.

    \item Visualize the following results with \textbf{\textit{figure}} and \textbf{\textit{imshow}}. 
    Describe each of the results with one sentence each. 

        \begin{itemize}
        \item $A$
        \item $B$
        \item Intersection between $A$ and $B$
        \item Union between $A$ and $B$
        \item Complement of intersection between $A$ and $B$
        \item Complement of union between $A$ and $B$
        \end{itemize}
\end{itemize}


\noindent
\newline
\textbf{2. Array Foray} Perform the following matrix operations. 
\begin{qparts}
    \item Use \textbf{\textit{reshape}} to create a $10 \times 10$ matrix $A$ where 
    $A = \begin{bmatrix}1 &11 & ...& 91\\ 2&12&...&92\\ \vdots&\vdots&\ddots&\vdots\\ 10&20&...&100\end{bmatrix}$.

    \item Use \textbf{\textit{magic}} to create a $10 \times 10$ magic matrix $B$. 
    Use $B$ to create a matrix $C$ which has the same diagonal values of B and is zero elsewhere. 
    \textbf{Note}: You might want to look up \textbf{\textit{diag}} to see how to do this elegantly. 

    \item Flip the second column of $B$ such that the column is inverted up down.

    \item Flip the matrix $A$ from left to right.

    \item Make \textit{cSum} the column-wise sum of every column of AB (normal matrix multiplication).
    The result should be a row vector.

    \item Make \textit{cMean} the row-wise mean of every row of AB (element-wise matrix multiplication).
    The result should be a column vector.

    \item Delete the last column of $A$. 
\end{qparts}


\end{document}
